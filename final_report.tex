\documentclass[fontsize=11pt]{article}  
\usepackage{amsmath}  
\usepackage[utf8]{inputenc}  
\usepackage[margin=0.75in]{geometry}  
\usepackage{hyperref}  
  
\title{  
The Effect of COVID 19 in Working Hours in Different Industries Across Canada}  
\author{Eren Aydin, Thomas Wu, Sheldon Dacon, Khizer Ahmad}  
\date{Tuesday, December 14, 2021}  
  
\begin{document}  
\maketitle  
  
\section*{Introduction}  
The global effects of COVID-19 does not need much introduction. Without a doubt, the pandemic has impacted every aspect of our lives to some level. It caused many changes in the dynamics of how the world operates; some for the better, some for the worse.   
  
One of the most affected markets was the job market. COVID changed a lot of things such as including how people work, what jobs they prefer, their work-life balance and so on. The goal of this project is to investigate what changes COVID-19 caused in the working hours of the industries in Canada. That is, how was the state of the working hours before the pandemic (defined as January-December 2019) and during the pandemic (defined as January 2020- December 2020).  
  
The question we seek to answer is:   
\textbf{``How has COVID-19 affected the working hours of specific Canadian industries?"}  
  
  
\section*{Dataset Description}  
We will be using 2 datasets for this project. Both of the datasets contain the monthly average of working hours (weekly) of each industry in Canada. The data is sourced from the Government of Canada, more specifically, statistics Canada. The data is in a CSV format. One of the datasets contains the data from 2019, and the other contains the same data from 2020. Both datasets also contain the monthly average of all of the industries combined. We will not be looking at the grand total, however, instead focusing on the industries individually.   
 
 

 
 
 % I need to find out if hours are self reported or not, government website doesn't give any info about where they get their data%
  
\section*{Computational Overview}  
The computations we plan to do and their explanations are as follows (They are in no particular order.):  
  
Our Project is divided into 3 python files that each have a specfic function.

\medskip   

\textbf{project\_part 1:} The purpose here is to plot the graph according to the selected industry regarding the data from before the pandemic (2019) and during the pandemic (2020) the two year will correspond o 2 different graphs and the x-axis will represent the average working hour while the y-axis
will be representing in the form of year and month. By creating line graphs for each industry before and during the pandemic, we illustrate their general pattern, i.e  much they tend to change over the course of each month. 

\medskip 


first, the function will use the pandas library and use the read\_csv method to create the DataFrame from the csv dataset. our program then stores this in
two variables.
We then created dictionary that map each industry  to the following data set from the DataFrame.
 % is this right?%

we then created 2 small helper functions called points\_of\_during\_pandemic
and points\_of\_pre\_pandemic. They work by creating the list of points from the dataset based on the chosen industry's average working hours and month before the pandemic for \_of\_pre\_pandemic or after the pandemic for points\_of\_during\_pandemic

Our main function is called plotting\_the\_graph. Its a rather complex algorithm that incorporates 3 for loops and many functions from the matplotlib that creates a graph based on the industry chosen. Unlike many simpler implementations of graphs using mathplotlib, in the function, the name will be changed so that all word will capitalize their first letter for every industry to make the output look more polished and professional.

This function works by first Getting the list of coordinates of the chosen industry by using helper functions. The the If the first alphabet of the word is lower cased. If it is, then the for loop changes it to upper case,the first word of the industry should always have an uppercase letter.

%Im gonna skip taking about the accumulators%

then this function uses matplotlib functions to set up the correct parameters for the graph by using the mathplotlib.plot, mathplotlib.sub\_plot, mathplotlib.setgp and more, we can create a graph with labels that start with a uppercase letter, thanks to the for loop that we created

\medskip 

\textbf{project\_part 2.}

In this part created a function that created doing tables showing how much the working hours changed both in numbers and in percentages. For example, a table will the working hours of an industry for January 2019 and January 2020, and then will include how much the working hours changed in numbers and in percentages. This will allow us to make an estimate of how much popularity the industry gained or lost with the introduction of the pandemic,

We started by importing the dictionary we created in part 1

first, the function will use the pandas library and use the read\_csv method to create the DataFrame from the csv dataset. our program then stores this in
two variables.
We then created dictionary that map each industry  to the following data set from the DataFrame.

We then created a function called create\_dataframe which creates a datadrame using the imported dictionary. the DataFrame where rows and columns are 0-indexed by default. we the use the same for loop from part in which the first letters are all capitalised.

(will add more I'm thinking about shortening this part)



\medskip 

\textbf{project\_part 3.}

in part 3.

the goal of part 3 is to Calculating the differences of the yearly averages of working hours for each industry pre pandemic and during the pandemic. This will allow us to further illustrate the sectors that gained or lost popularity, or stated relatively neutral. If our result is positive, we can say that the industry gained popularity with how much it gained popularity depending on the magnitude of our result, and vice versa.
  
 (will add code analysis later)
  
\section*{Instructions} 
 
\textbf{step 1.}
download all of the libraries under requirements.txt. These libraries are required in order for our functions to run normally.
 
\textbf{step 2.}
download the 2 datasets from the government of Canada from the url in markus, be sure to save them in a separate folder and \textbf{do NOT change the name of the CSV file}. If you do change the names of the files, the function will not work.
 
\textbf{step 3.} run the file called main.py. 
you should see and interactive screen with buttons that you can push, and interact with. Should should not be receiving any errors, if you do get an error, check to make sure you did not change the names of the CSV files.
 
\section*{Changes Made From Feedback}  

We received mostly positive feedback from the TAs after we submitted our proposal. aside from from minor formatting issues, the main concern of the TAs was the the scope of our topic. Our original idea was more focused on the change in the work hours of all Canadian industries.The TAs suggested that we focus more on the changes in one industry then all of them at once. They also wanted us to give more detail on where the data came from specifically how the data was measured.
 
 
 
 
 
\section*{Discussion} 

The functions that we created helped to further explore various modules as well as provide insights into the affects of Covid-19 using empirical data. By implementing functions in our project, we have accurately answered our original question on How has COVID-19 affected the working hours of specific Canadian industries? In part 1 of our project the functions we created 2 graphs according to the selected industry regarding the data
from before the pandemic (2019) and during the pandemic (2020) with the x-axis will represent the average working hour while the y-axis
will be representing in the form of year and month. Through the data we saw that for nearly every month in 2020, the reported working hours were less then the year before, which was before the global pandemic. This trend is very obvious as shown in our interactive GUI. Some industries show a greater change in hours than others by simply looking at the height differences in the bars. Some examples include accommodation and food services. In part 2 and 3 we expanded upon this idea and included functions that would calculate the percentage difference in the average working hours during and before the pandemic. This percentage difference clearly demonstrated a change between the two years which is solid proof that the pandemic did affect the report work hours.  (might add some more?)

We did run into some minor limitations specifically with calculating the percentage change in different industries. we found it very difficult to create a table using mathplotlib





\section{References}
 
\href{ https://www150.statcan.gc.ca/t1/tbl1/en/tv.action?pid=1410003601 }{ Government of Canada, Statistics Canada. Actual Hours Worked by Industry, Monthly, Unadjusted for Seasonality, Government of Canada, Statistics Canada, 8 Oct. 2021, https://www150.statcan.gc.ca/t1/tbl1/en/tv.action?pid=1410003601. } 

  
  
% NOTE: LaTeX does have a built-in way of generating references automatically,  
% but it's a bit tricky to use so we STRONGLY recommend writing your references  
% manually, using a standard academic format like APA or MLA.  
% (E.g., https://owl.purdue.edu/owl/research_and_citation/apa_style/apa_formatting_and_style_guide/general_format.html)  
  
\end{document}  
About MarkUs
